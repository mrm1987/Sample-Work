\documentclass{article}
\usepackage[top=.6in, bottom=.6in, left=.6in, right=.6in, a4paper, landscape]{geometry}
\usepackage{textcomp}
\usepackage[pdftex]{graphicx}
\graphicspath{{/home/matt/School/Pragmatics}}
\usepackage{enumitem}
\setlist[description]{leftmargin=8mm,labelindent=8mm}
\setlist[itemize]{leftmargin=8mm}
\usepackage{sectsty}
\subsectionfont{\normalfont\itshape}
\usepackage{multirow}
\setlength{\columnsep}{10mm}
\usepackage{fix2col}
\usepackage{multicol}

\begin{document}
\begin{multicols}{2}
	\begin{center}
		{\LARGE Ling 317 - Common Ground}\\\vspace{3mm}
		{\large Matthew Myers \& Sally Hausken\\\vspace{1.5mm}
		May 11, 2016}
	\end{center}
\tableofcontents
\section{Shared Knowledge (Horton, 2012)}
	\textbf{Main points}
	\begin{itemize}
		\item Communication is shaped in large part by contextual information regarding (supposed) knowledge of one's conversational partners.
		\item The complexity involved with navigating such a complex system is nontrivial.
		\item An understanding of successful communication will need to identify the particular psychological processes which shape common ground.  
	\end{itemize}
\subsection{The foundations}
To adhere to Grice's Cooperative Principle, a speaker must have some knowledge about their conversational partner and what they are capable of inferring.\\\vspace{-4mm}
\begin{description}
	\item[(1)] \textbf{Underspecify:} \#Did you hear? They all died! \vspace{-2mm}
	\item[(2)] \textbf{Overspecify:} \textit{To Grandma}: \#Dillo Day was a blast.
\end{description}
A speaker must also have some idea of what the addressee believes the speaker to know.

\begin{description}
	\item[(3)] \textit{Person A}: Oh first of all, I have Shana's shower coming up.\\\vspace{-5.0mm}
	
	\hspace{6.6mm}\textit{Person B}: Have you heard her latest news?\\
	\mbox{}\hfill (Horton, 2012) 
\end{description}
\noindent Lewis's coordination problem
\begin{itemize}
	\item The goal is for two agents to reach a mutually accepted outcome, usually via implicit means. \vspace{-1mm}
	\item To optimize your chances at meeting at the same restaurant as your friend, you should consider your friend's perspective.  
\end{itemize}
\begin{center}
\begin{tabular}{cc|c|c|c|}
	\cline{3-4}
	& & \multicolumn{2}{ c| }{Person A}\\ \cline{3-4}
	& & Luigi's & Fabio's \\ \cline{1-4}
	\multicolumn{1}{ |c  }{\multirow{2}{*}{Person B} } &
	\multicolumn{1}{ |c| }{Luigi's} & 1,1 & 0,0 \\ \cline{2-4}
	\multicolumn{1}{ |c  }{}		&
	\multicolumn{1}{ |c| }{Fabio's} & 0,0 & 1,1 \\ \cline{1-4}
\end{tabular}\\\vspace{2.5mm}
\hfill (Stanford Encyclopedia of Philosophy)
\end{center}
\begin{itemize}
	\item Precedents may guide the decision, but those will be further driven by expectations of precedence or \textit{common knowledge}.
\end{itemize}
\subsection{The mutual knowledge paradox}
Common knowledge relies on \textquoteleft beliefs about beliefs'.\\\vspace{-2mm}

\noindent\textbf{Problem: }Definitions results in an infinite regress.\\

A and B mutually know that \textit{p} iff \vspace{-1.5mm}
\begin{description}
	\item[1.] A knows that \textit{p}\vspace{-1.5mm}
	\item[1$^{\prime}$.] B knows that \textit{p}\vspace{-1.5mm}
	\item[2.] A knows that B knows that \textit{p}\vspace{-1.5mm}
	\item[2$^{\prime}$.] B knows that A knows that \textit{p} \vspace{-1.5mm}
	\item[3.] A knows that B knows that A knows that \textit{p} \vspace{-1.5mm}
	\item[3$^{\prime}$.] B knows that A knows that B knows that \textit{p}\vspace{-5mm}\\
	
	\hspace{6mm}\mbox{}\ldots\ldots\ldots\ldots\\
\end{description}\vspace{-2mm}
\hfill (Horton, 2012)
\subsection{Psychology of mutual knowledge}
\textbf{Sperber and Wilson}
\begin{itemize}
	\item Knowledge can be possessed by virtue of one's \textit{cognitive environment} or their \textquotedblleft physical surroundings and cognitive abilities"
	\begin{description}
		\item[(4)] Napoleon never played baseball.
	\end{description}
	\item This weaker sense of \textquoteleft know' constitutes what's \textit{mutually manifest}.
\end{itemize}

\indent How do assumptions of \textit{mutually manifestness} resolve?
\begin{itemize}
	\item \textbf{Principle of Relevance:} \textquotedblleft\dots every utterance communicates a presumption of its own optimal relevance. The relevance of utterances is determined in part by \ldots the impact [they] have on an addressee's cognitive environment."
	\begin{itemize}
		\item \textbf{Discussion:} Does this get around the mutual knowledge paradox? Is this theory amenable to empirical analysis?
	\end{itemize}
\end{itemize}\vspace{3mm}

\flushleft\textbf{Clark and Marshall}
\begin{itemize}
	\item Mutual knowledge is grounded by \textit{copresence heuristics}. 
	\begin{itemize}
		\item When there is evidence for \textit{triple copresence} (i.e., speaker, addressee, and referent are \textquotedblleft openly present") mutual knowledge can be inferred.
		\item \textbf{Copresence Types:} physical, linguistic, and community copresence.
	\end{itemize}
	\item \textbf{Common Ground:} Knowledge believed to be shared by conversational participants.
	\begin{itemize}
		\item Contains \textit{personal common ground} (inferred from physical and linguistic copresence) and \textit{communal common ground} (inferred from community copresence).
	\end{itemize}
\end{itemize}
\subsection{Common ground in action}
Adjustments to common ground is and dynamic process and subject to change as communication unfolds.
\begin{itemize}
	\item Presuppositions are propositions whose truths are taken for granted (Stalnaker). 
	\item Speakers necessarily believe that presuppositions are shared, but beliefs about commonality of presuppositions need not align.
	\begin{description}
		\item[(5)] I have to take my dog to the vet.
	\end{description}
	\item \textbf{Presupposition Accommodation:} Information that is advanced to common ground based on others' belief that it is shared.
	\item \textbf{Grounding:} Mutual effort to establish a set of common beliefs.
	\begin{itemize}
		\item Communication is a joint action with the general purpose of satisfying some communicative goal.
		\item \textbf{Grounding Criteria:} Participants will work to meet some sufficient degree of understanding particular to the context.
	\end{itemize} 
\end{itemize}
\textbf{Clark \& Wilkes-Gibbs (1986)} Referential Communication
\begin{itemize}
	\item Participants asked to sort abstract Tangrams.
	\item Tentative descriptions (e.g., hedges, intonational contours, etc.) invited further feedback. 
\end{itemize}
\begin{center}
\includegraphics[scale=.4]{tangram}
\end{center}
\begin{itemize}
	\item \textbf{Conceptual Pacts:} Tacit agreement between partners regarding referents (e.g., \textquotedblleft The one that kinda looks like a guy slouching" $\longrightarrow$ \textquotedblleft The slouching guy")
	\item \textbf{Principle of Least Collaborative Effort:} Conversational partners will work to minimize joint effort in communicating.
\end{itemize}
People revert to more basic-level descriptions when communicating with new partners (Brennan \& Clark, 1996)
\begin{itemize}
	\item \textbf{Discussion:} It's noted that such behavior is a violation of Grice's Maxim of Quantity (p.388). But Grice's maxim states information should be \textquotedblleft\ldots neither more or less than is required" (Grice, 1975). In what way does this finding contradict Grice's maxim?
\end{itemize}
\subsection{Constraints on pragmatic processing}
\textbf{Full Constraints:} When processing an utterance, listeners can limit memory access to that information which is presumed to be common ground between the speaker and addressee. 
\begin{itemize}
	\item \textbf{Principle of Optimal Design:} Belief that speaker will design utterance to be properly understood given the common ground (aka, recipient design, audience design, and communication accommodation)
	\item \textbf{Clark, Schreuder, \& Buttrick (1983)} Flower Prominence
	\begin{itemize}
		\item Asked participants, \textquotedblleft What is the color of this flower?"\\ \vspace{3mm}
		\hspace{2mm}\includegraphics[scale=.45]{flower}
		\item The color of the prominent flower was more common than \textquotedblleft Which flower?"
	\end{itemize} 
\end{itemize}
\textbf{Individual Constraints:} The degree to which utterance processing is constrained by individuals' limitations.
\begin{itemize}
	\item \textit{Privileged information}\textemdash information only known to the speaker\textemdash has been shown to be acted upon in certain conversational situations.
	\item \textbf{Keysar et al. (2000)} Privileged Candles
	\begin{itemize}
		\item A confederate (i.e., the director) was instructed to ask a real participants (i.e., the addressee) to rearrange objects on a shelf.
		\item For this example, the critical item was the \textquotedblleft small candle".\\\vspace{2mm}
		\hspace{10mm}\includegraphics[scale=.3]{Box}
		\item Participants initially began moving towards the privileged candle more often than either of the other candles.
		\item This is taken as evidence for a \textit{perspective adjustment} account of common ground processing.
	\end{itemize}
\end{itemize}
\textbf{Probabilistic Constraints:} The salience and strength of available cues will determine what people infer their partner's beliefs.
\begin{itemize}
	\item \textbf{Hanna, Tanenhaus, \& Trueswell (2003)} Privileged Triangles
	\begin{itemize}
		\item Similar to Keysar et al. (2000), but the unprivileged competitor was identical with the common ground item.
		\item Additionally, competitors also varied in an additional dimension: color.\\\vspace{2mm}
		\hspace{10.5mm}\includegraphics[scale=.4]{triangle}
		\item Findings: (a) In the privileged condition and when both triangles are in common ground, there were \textit{less} looks at the competitor than target, but \textit{more} looks at the competitor than unrelated items. And (b) they were faster to begin looking at the target when they were both in common ground.
	\end{itemize} 
\end{itemize}
\subsection{Interactive Influences}
People aim to consider common ground, but performance may vary by imposed contextual or individual constraints.
\begin{itemize}
	\item \textbf{Horton \& Keysar (1996)} Common Ground Under Pressure
	\begin{itemize}
		\item There were two conditions: \textquotedblleft shared context" and \textquotedblleft privileged context". Each condition was also either speeded or unspeeded.
		\item For each condition, the speaker was told to describe the moving object such that the listener could identify it.\\	
	\end{itemize}
	\hspace{4mm}\includegraphics[scale=.4]{Horton1}\includegraphics[scale=.41]{Horton2}
	\begin{itemize}
		\item The descriptions used by the speaker were recored as a proportion of the context-related (e.g.,\textquotedblleft big") versus context-independent (i.e., \textquotedblleft centered") descriptions.
	\end{itemize}
	\hspace{12mm}\includegraphics[scale=.5]{Horton3}
\end{itemize}
\textbf{Interactive Alignment Account:} Mutual knowledge is derived via domain-general cognitive mechanisms (such as unconscious priming).
\begin{itemize}
	\item \textbf{Cleland and Pickering (2003)} Semantic Syntactic Alignment
	\begin{itemize}
		\item After hearing, \textit{the car that's blue}, people are more likely to say \textit{the sheep that's red} than \textit{the red sheep}.
		\item Furthermore, people were more likely to say \textit{the goat that's red} after hearing \textit{the sheep that's red} than \textit{the car that's red}.
	\end{itemize}
\end{itemize}
\textbf{Memory-Based Account:} Consideration of partner's knowledge can be reached through automatic \textit{or} strategic means.
\begin{description}
	\item[(6)] Yeah, I've got another buddy who, uh, is a Marine pilot. I'm trying to think if you have ever met this guy. I don't think so.\\
	\hfill (Horton \& Gerrig, 2005)
\end{description}
\begin{itemize}
	\item However, strategic considerations may be uncommon compared to automatic processes.
	\item \textbf{Resonance:} An automatic process of retrieval where contextual cues and long-term memory interact to aid in memory accessibility. 
\end{itemize}
\section{Goals on Reference Production (Yoon et al., 2012)}

\textbf{Main Points}
\begin{itemize}
	\item To what degree do speakers take into account the addressee’s perspective?
	\item Do speakers design expressions with respect to Common Ground?
\end{itemize}
\subsection{Perspective in Production}
When do speakers take into account the addressee's perspective when designing referential expressions?\\

\textbf{Two prominent views:}
\begin{enumerate}
	\item \textbf{Constraint-based Processing View:} Multiple sources of information, including contextual information and the perspective of the addressee, combine to constrain language comprehension and production processes (Hanna, Tanenhaus \& Trueswell,2003; Horton,2007; Tanenhaus \& Trueswell,1995) predicts early and strong contributions of common ground to referential production in cases where this information is well-established and relevant.
	\item \textbf{Egocentric-heuristic View:} Proposes that expressions are initially designed from the speaker's egocentric perspective (Horton \& Keysar,1996)\textemdash meaning that the speaker monitors the addressee for understanding, and in cases of confusion, adjusts the expression to meet the addressee’s needs.
\end{enumerate}
\textbf{Previous Research:}
	\begin{itemize}
	\item Little experimental research has been done on common ground. And what has been done focuses on American English.
	\item Some research has suggested that the addressee's perspective is only relevant during a delayed, second stage of production.
	\begin{description}
		\item[Horton \& Keysar 1996:] speakers described a moving target object so that the addressee could determine whether they saw the same object move. One group was given a time crunch and the others not. The group that was not under time pressure were more likely to give contextually relevant information (e.g., \textquotedblleft small") when the target object was in the common ground. This is consistent with the claim that use of perspective is time-consuming. 
		\item[Wardlow Lane, Groisman, and Ferreira 2006:] Speakers referred to a mutually known object in a context that contained a size-contrasting object in the speaker’s privileged ground and describe it so that the addressee could identify it. In one group, speakers were to conceal the identity of the privileged object. The prediction was that speakers would be less likely to use a scalar modifier in this context, because scalar modification implies contrast, but actually the use of scalar modifiers increased, suggesting that privileged information is automatically incorporated into referential processes, and thus is outside the speaker's control.
	\end{description}
	\item Other studies have shown moderate effects of perspective on production.
	\begin{description}
		\item[Nadig \& Sedivy 2002:] 5-to 6-year-old children (often considered egocentric), showed successful use of common ground in both production and comprehension in situations in which the communicative goal was to request that the communicative partner pick up an object.
		\item[Heller et all 2012:] Speakers sometimes referred to privileged information when making requests, but they successfully distinguished privileged information from common ground in the same utterances. 
	\end{description}
	\item Previous research on the role of perspective in production has largely been limited to Americans, which show deficits in recognizing common ground in language production. But new research suggests that this may not be the case in other cultures. 
	\begin{description}
		\item[Wu \& Keysar 2007:] Chinese participants were highly sensitive to common ground during language comprehension.
	\end{description}
	\begin{itemize}
		\item Argues that people from individualist cultures (e.g., the United States) are more egocentric than those from interdependent cultures (e.g., China).
		\item Thoughts?
	\end{itemize}
	\item The speaker's communicative goals in producing an utterance are an essential component of language production. However, little to no research has examined how goals affect the use of perspective in production.
	\begin{itemize}
		\item In the previous research on audience design, the speaker's communicative goal was not considered or manipulated
		\item In studies that found limitations on audience design in adults, the speaker's goal was to describe for the purpose of the addressee identifying a referent (Horton \& Keysar, 1996; Wardlow Lane et al.,2006).
		\item Informing vs. Requesting: different demands on perspective taking?
	\end{itemize}
	\end{itemize}
\subsection{Experiment 1}
\begin{description}
	\item[Goal:] Examine whether Korean speakers would use scalar-modified noun phrases to identify objects in contexts in which the modifiers were not informative to the addressee.
	\item[Hypothesis:] If previous findings in an American sample (Horton \& Keysar,1996) extended to a Korean sample, then the speakers would produce scalars more often when modification would be informative to the addressee, and that there would be partial interference from the speaker's privileged perspective.
	\item[Participants:] 18 undergraduate students at Seoul National University, all native Korean speakers.
	\item[Materials:] 78 objects grouped into six sets of 13 objects each, with three critical pairs of objects per set (i.e., three targets and three competitors), plus seven distractor objects. The participants completed 18 critical and 18 additional, filler trials.\\\vspace{4mm}
	
	The number of competitor objects and whether they were mutually visible to the speaker and addressee were manipulated in three conditions:
	\begin{enumerate}
		\item One object condition: One target object. (e.g., A cup)
		\item Two-Object condition: here were two mutually visible critical objects, a target and a competitor (e.g., a small candle and a big candle). (The competitor was always the same type of object, but larger than the target object.)
		\item Two-Object-Privileged condition: the target was shared and the competitor was privileged (e.g., a small plate and a big plate). The speaker saw both the target and the competitor, but a curtain kept the competitor from the addressee's view.
	\end{enumerate}
	\item[Procedure:] The participant and the confederate sat on either side of a 5 × 5 grid with objects in some slots. Two notebook computers, one assigned to a participant and one to the confederate, displayed instruction pictures during the task. Four slots were blocked with a curtain on one side of the grid. In the two-object-privileged condition, the curtain blocked the competitor object from the addressee's view.
	\includegraphics[scale=.35]{Yoon}
	
	At the beginning of each set, the participant was asked to place the objects to duplicate a scene presented on the laptop, while the confederate turned away so they would not see the objects.\\\vspace{4mm}
	
	After the participant had placed the objects, the experimenter blocked four slots with black curtains. The confederate then faced the participant, and the experimenter asked the confederate to name each object that could be seen in the grid. The confederate then named each object using a bare noun (to avoid influencing the participant's productions). This was done to make it clear to the participants that they and their partner (the confederate) had different perspectives, and to establish which objects were common ground.\\\vspace{4mm}
	
	Each took turns asking the other to pick up objects from the grid. The confederate always referred to filler objects and did not use any adjectives.\\\vspace{4mm}
	
	The speakers' words were recorded and transcribed and the target object descriptions were categorized into three kinds: bare names (e.g.,\textquotedblleft Pick up the plate, please."), scalar-modified expressions (e.g., \textquotedblleft Pick up the small plate, please."), and errors, in which the participant asked his or her partner to pick up the wrong object.
	\item[Results:]When the partners had common ground, the participants were highly sensitive to referential context. Speakers typically produced a bare noun when there was only one object (e.g., plate, 93.52 \%) but produced a scalar-modified noun phrase when two different-sized objects of the same kind were in common ground (e.g., small plate, 73.15 \%). In the privileged round condition, the participants used a bare noun (50.00 \%) or a scalar-modified expression (46.30 \%) roughly equally, suggesting that both common ground and privileged ground influenced their utterance designs.\\\vspace{3mm}
	
	\includegraphics[scale=.75]{Yoon1}
	
	This experiment replicates the previous findings of moderate audience design in a new participant population and language (Korean). The results of Experiment 1 suggest that in language production, interference from one's own privileged knowledge may reflect a standard, rather than a culturally determined (e.g., Wu \& Keysar,2007), component of language production.
\end{description}
\subsection{Experiment 2}
\begin{description}
	\item[Goal:] Examine whether a speaker's sensitivity to the perspective status of competitor objects is influenced by utterance goals, specifically informing and requesting.
	\item[Hypothesis] Speakers would be more likely to design expressions with respect to common ground when requesting, as the addressee must interpret the speaker's message precisely in order to complete the request. In contrast, interpretation of information in the absence of some explicit behavioral goal (e.g., to move something) may afford less precise-that is, \textquotedblleft good enough"-understanding (e.g., Ferreira, Bailey \& Ferraro,2002).
	\item[Participants:] 30 pairs of students at Seoul National University, who had not participated in Experiment 1.
	\item[Materials:] Similar to Experiment 1. Each participant received 11 objects per set, with six critical objects (three targets and three competitors), plus five distractors. The participants were presented with 18 critical and 18 filler trials. Unlike in Experiment1, the target was the larger of the two competitor objects on half of the trials, and the smaller on the other half. Pairs of critical objects were placed near each other so that participants could easily identify the two objects and appreciate that their perspective conflicted with their partner's. The critical objects were rotated across the critical conditions on three lists.
	\item[Procedure:] Two manipulations were used: common ground (three conditions, identical to those of Exp. 1) and utterance goal (informing vs. requesting). On informing trials, the experimenter moved the objects,and the speaker provided information to the addressee about where the objects would be moved before the experimenter moved them (e.g., \textquotedblleft The experimenter will move the plate to the left"). On requesting trials, the speaker asked the addressee to move objects (e.g., \textquotedblleft Can you move the plate to the left?"), and the addressee, moved the object. The common-ground conditions were manipulated within subjects, and utterance goal (request vs. inform) was manipulated between subjects. The speakers and listeners continuously played their respective roles throughout the entire experiment. 
	\item[Results:]When perspectives differed (two-object privileged condition), the utterance goal modulated referring, with modification rates of 36.67 \% and 60.00 \% in the requesting and informing conditions, respectively.\\\vspace{4mm}
	
	The results replicated those of Experiment1, with significantly lower modification rates in the one-object condition than in the two-object and two-object-privileged conditions, and less modification in the two-object-privileged condition than in the two-object condition.\\\vspace{4mm}
	
	Suggests that when speakers were in a situation in which they had to ask for something, they were more likely to avoid unnecessary and potentially confusing adjectives.\\\vspace{3mm}
	\includegraphics[scale=.5]{Yoon2}\\
	Since the two participants were facing each other, the speaker's egocentric right was the addressee's egocentric left. Thus, when requesting, speakers always adopted the addressee's perspective, using expressions such as “on your left” when giving directions, whereas when informing, five of the 15 participants consistently used egocentric spatial terms designed from their own perspective. Thus, tailoring the utterance to the addressee's point of view may depend on what the speaker is trying to accomplish.
	
	\includegraphics[scale=.7]{Yoon3}
\end{description}
\subsection{Discussion}
The results of experiment 1 show that there is a sensitivity to common ground in utterance design, as demonstrated by the fact that adjectives were used significantly less often when the competitor was privileged.  But adjectives were used that were not necessary on half the trials, showing that information from both the speaker's and addressee's perspectives influences production. This is particularly useful because it examines a new culture and language that had not been researched before.\\\vspace{4mm}

\textit{Why did Korean participants fail to completely suppress privileged objects, unlike the Chinese in Wu and Keysar's sample?}
\begin{itemize}
	\item One possibility: In language production (unlike comprehension), the speaker's privileged perspective is always relevant, as speaking often involves imparting new, previously privileged information to the addressee.
	\item Other possibilities?
\end{itemize}
Experiment 2 provides new, key evidence that utterance goals influence a speaker's sensitivity to perspective.
\begin{itemize}
	\item When critical objects were in common ground, goals played little role in referring, and the results were similar to those of Experiment 1. But when the competitor was privileged, the percentage of bare-noun responses was significantly higher when speakers were requesting versus informing.
\end{itemize}
\textit{Why were speakers more sensitive to perspective when requesting?}
\begin{itemize}
	\item When one's goal is to make a request, it is more critical that the addressee interpret the speaker's message precisely, in order for the requested action to be carried out correctly.
\end{itemize}
\subsection{Conclusions}
\textbf{Two important conclusions:}
\begin{enumerate}
	\item The addressee's ability to uniquely identify the intended referent is not necessary for communicative success. Instead, communicative success must be interpreted with respect to the speaker's goals.
	\item Failures of the addressee may say more about whether the listener's unique perspective is, in fact, relevant to the communicative goals, rather than about the ability of speakers to engage in audience design.
\end{enumerate}
\end{multicols}
\end{document}